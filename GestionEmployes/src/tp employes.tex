\documentclass[a4paper,12pt]{report}

\usepackage[utf8]{inputenc}  % Support UTF-8
\usepackage[T1]{fontenc}     % Codage des caractères
\usepackage[french]{babel}   % Langue française
\usepackage{geometry}        % Mise en page
\usepackage{graphicx}        % Images

% Début du document
\begin{document}

% Page de titre
\title{\textbf{Compte Rendu du Projet de Gestion des Employés en Java}}
\author{Réalisé par : Nouhaila GHANDOUR}
\date{\today}  % Date actuelle
\maketitle

% Contenu du document
\chapter*{Introduction}
Ce document présente le projet de gestion des employés réalisé en Java, utilisant une architecture Modèle-Vue-Contrôleur (MVC). Ce projet permet d'ajouter, modifier, supprimer et afficher des informations concernant les employés d'une entreprise. L'application interagit avec une base de données MySQL pour stocker les informations des employés.

\chapter{Description du Projet}
\section{Objectifs Principaux}
L'objectif du projet est de développer une application qui permet de :
\begin{itemize}
    \item Ajouter, modifier, supprimer et afficher des employés.
    \item Valider les informations des employés (email, téléphone, salaire).
    \item Gérer les rôles et les postes des employés.
\end{itemize}

\section{Technologies Utilisées}
Le projet utilise les technologies suivantes :
\begin{itemize}
    \item **Java** : Langage de programmation principal pour le développement de l'application.
    \item **Architecture MVC** : Modèle-Vue-Contrôleur pour séparer la logique métier, la présentation et les contrôles.
    \item **MySQL** : Base de données utilisée pour stocker les informations des employés.
\end{itemize}

\chapter{Explication du Code}
\section{Architecture MVC}
L'application suit le modèle d'architecture \textbf{MVC} qui sépare les responsabilités de l'application en trois parties :
\begin{itemize}
    \item \textbf{Modèle} : Contient la logique métier et l'accès aux données.
    \item \textbf{Vue} : Représente l'interface utilisateur. Elle permet d'afficher les informations et d'interagir avec l'utilisateur.
    \item \textbf{Contrôleur} : Gère l'interaction entre la vue et le modèle. Il reçoit les actions de l'utilisateur, les traite et met à jour la vue ou le modèle en conséquence.
\end{itemize}

\section{Le Modèle}
Le modèle contient la logique métier de l'application, notamment les méthodes pour ajouter, modifier, supprimer et afficher les employés.

\subsection{Classe EmployeModel}
La classe \texttt{EmployeModel} est responsable de la gestion des données des employés. Elle interagit avec le \texttt{EmployeDAOimpl} pour effectuer des opérations sur la base de données.

\subsubsection{Méthode \texttt{ajouterEmploye}}
La méthode \texttt{ajouterEmploye} ajoute un employé à la base de données après avoir validé les informations (email, téléphone et salaire). Voici son code :
\begin{verbatim}
public boolean ajouterEmploye(int id, String nom, String prenom, String email,
        String telephone, double salaire, Role role, Poste poste) {
    if(email == null || !email.contains("@")) {
        System.out.println("Erreur: Email doit contenir @");
        return false;
    }
    if(salaire <= 0) {
        System.out.println("Erreur: Salaire invalide");
        return false;
    }
    if (!telephone.matches("\\d{10}")) { 
        System.out.println("Numéro de téléphone invalide : " + telephone);
        return false;
    }
    Employe nouvelEmploye = new Employe(id, nom, prenom, email, telephone, salaire, role, poste);
    dao.ajouterEmploye(nouvelEmploye);
    return true;
}
\end{verbatim}
Cette méthode vérifie la validité des informations fournies. Si elles sont correctes, elle crée un nouvel objet employé et le passe à la méthode \texttt{ajouterEmploye} du DAO pour l'enregistrer dans la base de données.

\subsubsection{Méthode \texttt{modifierEmploye}}
La méthode \texttt{modifierEmploye} permet de modifier les informations d'un employé existant. Elle vérifie également la validité des informations avant de procéder à la mise à jour dans la base de données :
\begin{verbatim}
public boolean modifierEmploye(int id, String nom, String prenom, String email, String telephone, double salaire, Role role, Poste poste) {
    if(email == null || !email.contains("@")) {
        System.out.println("Erreur: Email doit contenir @");
        return false;
    }
    if(salaire <= 0) {
        System.out.println("Erreur: Salaire invalide");
        return false;
    }
    if (!telephone.matches("\\d{10}")) { 
        System.out.println("Numéro de téléphone invalide : " + telephone);
        return false;
    }
    Employe employe = new Employe(id, nom, prenom, email, telephone, salaire, role, poste);
    dao.modifierEmploye(id, employe);
    return true;
}
\end{verbatim}
Cette méthode met à jour les informations de l'employé spécifié dans la base de données via la méthode \texttt{modifierEmploye} du DAO.

\subsubsection{Méthode \texttt{supprimerEmploye}}
La méthode \texttt{supprimerEmploye} supprime un employé en vérifiant d'abord si l'ID est valide :
\begin{verbatim}
public boolean supprimerEmploye(int id) {
    if(id <= 0) {
        System.out.println("Erreur: ID invalide");
        return false;
    }
    return dao.supprimerEmploye(id);
}
\end{verbatim}
Si l'ID est valide, la méthode appelle la méthode \texttt{supprimerEmploye} du DAO pour supprimer l'employé de la base de données.

\subsubsection{Méthode \texttt{afficherEmployes}}
La méthode \texttt{afficherEmployes} récupère la liste des employés depuis la base de données pour les afficher dans l'interface utilisateur :
\begin{verbatim}
public List<Employe> afficherEmployes() {
    return dao.afficherEmployes();
}
\end{verbatim}

\section{Le Contrôleur}
Le contrôleur est responsable de la gestion de l'interaction entre la vue et le modèle. Il traite les actions de l'utilisateur et met à jour l'affichage en conséquence.

\subsection{Classe EmployeController}
La classe \texttt{EmployeController} contient les méthodes qui interagissent directement avec la vue. Elle récupère les entrées de l'utilisateur, appelle les méthodes appropriées du modèle et met à jour la vue en fonction des résultats.

\subsubsection{Méthode \texttt{ajouterEmploye}}
La méthode \texttt{ajouterEmploye} récupère les informations de l'utilisateur, appelle la méthode du modèle pour ajouter un employé, puis affiche un message de succès ou d'erreur selon le résultat :
\begin{verbatim}
public void ajouterEmploye() {
    String nom = view.getNom();
    String prenom = view.getPrenom();
    String email = view.getEmail();
    String telephone = view.getTelephone();
    double salaire = view.getSalaire();
    Role role = view.getRole();
    Poste poste = view.getPoste();

    boolean ajoutReussi = model.ajouterEmploye(0, nom, prenom, email, telephone, salaire, role, poste);

    if(ajoutReussi) {
        view.afficherMessageSucces("Employé ajouté avec succès.");
    } else {
        view.afficherMessageErreur("Échec de l'ajout.");
    }
}
\end{verbatim}

\subsubsection{Méthode \texttt{modifierEmploye}}
La méthode \texttt{modifierEmploye} permet de modifier les informations d'un employé. Elle récupère l'ID de l'employé et les nouvelles informations, puis appelle la méthode de modification du modèle et met à jour la vue :
\begin{verbatim}
public void modifierEmploye() {
    try {
        int id = view.getId();
        String nom = view.getNom();
        String prenom = view.getPrenom();
        String email = view.getEmail();
        String telephone = view.getTelephone();
        double salaire = view.getSalaire();
        Role role = view.getRole();
        Poste poste = view.getPoste();
        
        boolean modificationReussie = model.modifierEmploye(id, nom, prenom, email, telephone, salaire, role, poste);
        if (modificationReussie) {
            view.afficherMessageSucces("Employé modifié avec succès.");
            view.remplirTable(model.afficherEmployes());
        } else {
            view.afficherMessageErreur("Échec de la modification.");
        }
    } catch (NumberFormatException e) {
        view.afficherMessageErreur("ID ou salaire invalide !");
    }
}
\end{verbatim}

\subsubsection{Méthode \texttt{supprimerEmploye}}
La méthode \texttt{supprimerEmploye} permet de supprimer un employé à partir de son ID, puis met à jour la vue en affichant un message de succès ou d'erreur :
\begin{verbatim}
public void supprimerEmploye() {
    try {
        int id = view.getId();
        boolean suppressionReussie = model.supprimerEmploye(id);

        if(suppressionReussie) {
            view.afficherMessageSucces("Employé supprimé avec succès.");
            view.remplirTable(model.afficherEmployes());
        } else {
            view.afficherMessageErreur("Échec de la suppression.");
        }
    } catch (NumberFormatException e) {
        view.afficherMessageErreur("ID invalide !");
    }
}


\end{verbatim}



\section{Le DAO}
La classe DAO est responsable de la gestion des opérations de base de données. Elle interagit avec la base de données MySQL pour insérer, mettre à jour, supprimer et récupérer des employés.

\subsection{Classe EmployeDAOimpl}
Cette classe implémente les méthodes de la classe \texttt{EmployeDAO} et utilise JDBC pour interagir avec la base de données.

\chapter{Conclusion}
Le projet de gestion des employés en Java offre une solution simple et efficace pour gérer les employés d'une entreprise. Grâce à l'architecture MVC et l'intégration de la base de données MySQL, il est possible de réaliser des opérations de gestion d'employés tout en respectant les bonnes pratiques de développement.

\end{document}